\documentclass{article}
\usepackage{chemfig}
\usepackage{geometry}
\geometry{a4paper, landscape, margin=2cm}
\usepackage{xcolor}
\begin{document}
\definecolor{background}{HTML}{1e1e2e}
\definecolor{text}{HTML}{c6d0f5}
\pagecolor{background}
\color{text}
Branched molecule \vspace{.5cm}
\chemleft{[}
  \chemfig{L-[:30,,,,->]M(-[:150,,,,<-]L)(-[:30,,,,<-]L)(-[:90,,,,<-]L)(-[:-30,,,,<-]L)-[:-90,,,,<-]L}
\chemright{]^{x+}}
help\\
\chemfig{
    \charge{90=\:,145=-}{O}-[:-45]C(=[:-135]O)=C(=[:-45]O)-[:45]\charge{90=\:,35=-}{O}
}
sijos\\\\\\\\
\chemfig{
    H_2\charge{-90=\:}{N}-[:45]CH_2-CH_2-[:-45]\charge{-90=\:}{N}H_2
}
bhjkl acdefrsvwbhjklrfhjkb\\\\\\\\\\
\chemleft{[}
\chemfig{
    \charge{-30=\:}{Cl}-[:-30,,,,->]Pt(-[:30,,,,<-]\charge{-150=\:}{Cl})(-[:-30,,,,<-]\charge{150=\:}{N}H_3)(-[:-150,,,,<-]H_3\charge{30=\:}{N})
}
\chemright{]}\\\\
Wow isn't transition metals such a fun and not totally painful topic\\\\\\\\\\
\chemleft{[}
  \chemfig{H_3\charge{0=\:}{N}-[:0,,,,->]Ag -[:0,,,,<-]\charge{180=\:}{N}H_3}
\chemright{]^{+}}
\\\\ Wow you think tollen's is cool, wait till you see Haem\\\\
\chemleft{[}
  \chemfig{
  (globin protein)-[:90]\charge{90=\:}{N}-
  }
\chemright{]^{+}}
\\\\
\chemleft{[}
  \chemfig{L_1-[:30,,,,->]M(-[:150,,,,<-]L_1)(-[:30,,,,<-]L_2)(-[:90,,,,<-]L_2)(-[:-30,,,,<-]L_1)-[:-90,,,,<-]L_1}
\chemright{]}\\\\
\chemleft{[}
  \chemfig{L_1-[:30,,,,->]M(-[:150,,,,<-]L_1)(-[:30,,,,<-]L_1)(-[:90,,,,<-]L_2)(-[:-30,,,,<-]L_1)-[:-90,,,,<-]L_2}
\chemright{]}\\\\
\chemleft{[}
  \chemfig{L_1-[:30,,,,->]M(-[:150,,,,<-]L_2)(-[:30,,,,<-]L_1)(-[:-30,,,,<-]L_2)}
\chemright{]}\\\\
\chemleft{[}
  \chemfig{H_2\charge{30=\:}{O}-[:30,,,,->]M(-[:150,,,,<-]H_2\charge{-30=\:}{O})(-[:30,,,,<-]\charge{-150=\:}{O}H_2)(-[:90,,,,<-]\charge{-90=\:}{O}H_2)(-[:-30,,,,<-]\charge{150=\:}{O}H_2)-[:-90,,,,<-]\charge{90=\:}{O}H_2}
\chemright{]^{3+}}\\\\
\chemfig{
H-C(-H)(-H)-C(-H)(-H)-H
}\\\\
\newpage
We love mechanisms
\\ Nucleophilic Substitution\\\\
\schemestart
\chemfig{
R_1-@{a}\charge{45[anchor=180+\chargeangle]=$\delta^+$}{C}(-[:90]R_2)(-[:-90]R_3)-[@{b}]@{c}\charge{45[anchor=180+\chargeangle]=$\delta^-$}{X}
}\qquad
\chemfig{@{d}\charge{-120[anchor=180+\chargeangle]=\:}{Nu}}
\chemmove{
\draw[shorten <=2pt,shorten >=1pt](b).. controls +(90:5mm) and +(100:5mm)..(c);
\draw[shorten <=2pt,shorten >=1pt](d).. controls +(-120:10mm) and +(-30:10mm)..(a);
}
\arrow
\chemfig{R_1-C(-[:90]R_2)(-[:-90]R_3)-\charge{30:2pt=$+$}{Nu}}
\+
\chemfig{\charge{145:2pt=$-$,180=\:}{X}}
\schemestop\\\\

\schemestart
\chemfig{
R_1-@{a}\charge{45[anchor=180+\chargeangle]=$\delta^+$}{C}(-[:90]R_2)(-[:-90]R_3)-[@{b}]@{c}\charge{45[anchor=180+\chargeangle]=$\delta^-$}{X}
}\qquad
\chemfig{@{d}\charge{145:2pt=$-$,-120[anchor=180+\chargeangle]=\:}{C}N}
\chemmove{
\draw[shorten <=2pt,shorten >=1pt](b).. controls +(90:5mm) and +(100:5mm)..(c);
\draw[shorten <=2pt,shorten >=1pt](d).. controls +(-120:10mm) and +(-30:10mm)..(a);
}
\arrow
\chemfig{R_1-C(-[:90]R_2)(-[:-90]R_3)-C~N}
\+
\chemfig{\charge{145:2pt=$-$,180=\:}{X}}
\schemestop\\\\

\schemestart
\chemfig{
R_1-@{a}\charge{45[anchor=180+\chargeangle]=$\delta^+$}{C}(-[:90]R_2)(-[:-90]R_3)-[@{b}]@{c}\charge{45[anchor=180+\chargeangle]=$\delta^-$}{X}
}\qquad
\chemfig{@{d}\charge{-120[anchor=180+\chargeangle]=\:}{N}H_3}
\chemmove{
\draw[shorten <=2pt,shorten >=1pt](b).. controls +(90:5mm) and +(100:5mm)..(c);
\draw[shorten <=2pt,shorten >=1pt](d).. controls +(-120:10mm) and +(-30:10mm)..(a);
}
\arrow[-30]
\chemfig{R_1-C(-[:90]R_2)(-[:-90]R_3)-@{e}\charge{45:2pt=$+$}{N}(-[:90]H)(-[:-90]H)-[@{f}]@{g}H} \qquad
\chemfig{@{h}\charge{145=\:}{N}H_3}
\chemmove{
\draw[shorten <=2pt,shorten >=1pt](h).. controls +(145:5mm) and +(45:5mm)..(g);
\draw[shorten <=2pt,shorten >=1pt](f).. controls +(-90:4mm) and +(-45:4mm)..(e);
}
\+
\chemfig{\charge{145:2pt=$-$,180=\:}{X}}
\arrow[30]
\chemfig{R_1-C(-[:90]R_2)(-[:-90]R_3)-N(-[:60]H)(-[:-60]H)}
\+
\chemfig{NH_4^+}
\+
\chemfig{\charge{145:2pt=$-$,180=\:}{X}}
\schemestop\\\\
Elimination\\\\
\schemestart
\chemfig{R_1-C(-[@{a}:90]@{b}X)(-[:-90]R_2)-[@{c}]C(-[:90]R_3)(-[@{d}:-90]@{e}H)-R_4} \qquad
\chemfig{@{f}\charge{-145=\:}{A}}
\chemmove{
\draw[shorten <=2pt,shorten >=1pt](a).. controls +(0:5mm) and +(-30:5mm)..(b);
\draw[shorten <=2pt,shorten >=1pt](d).. controls +(180:4mm) and +(-90:4mm)..(c);
\draw[shorten <=2pt,shorten >=1pt](f).. controls +(-145:10mm) and +(0:10mm)..(e);
}
\arrow
\chemfig{R_1-[:-60]C(-[:-120]R_2)=@{z}C(-[:60]R_3)(-[:-60]R_4)}
\+{,,-26pt}
\chemfig[baseline=26pt]{HA^+}
\+{,,-26pt}
\chemfig[baseline=26pt]{\charge{145:2pt=$-$,180=\:}{X}}
\schemestop
\\\\

\schemestart
\chemfig{R_1-C(-[@{a}:90]@{b}X)(-[:-90]R_2)-[@{c}]C(-[:90]R_3)(-[@{d}:-90]@{e}H)-R_4} \qquad
\chemfig{@{f}\charge{145:2pt=$-$,-145=\:}{O}H}
\chemmove{
\draw[shorten <=2pt,shorten >=1pt](a).. controls +(0:5mm) and +(-30:5mm)..(b);
\draw[shorten <=2pt,shorten >=1pt](d).. controls +(180:4mm) and +(-90:4mm)..(c);
\draw[shorten <=2pt,shorten >=1pt](f).. controls +(-145:10mm) and +(0:10mm)..(e);
}
\arrow
\chemfig{R_1-[:-60]C(-[:-120]R_2)=@{z}C(-[:60]R_3)(-[:-60]R_4)}
\+{,,-26pt}
\chemfig[baseline=26pt]{H_2O}
\+{,,-26pt}
\chemfig[baseline=26pt]{\charge{145:2pt=$-$,180=\:}{X}}
\schemestop\\\\
Electrophilic addition anyone??\\\\

\schemestart
\chemfig{@{cc}{R_1}-[:-60]C(-[:-120]R_2)=[@{z}]C(-[:60]R_3)-[:-60]R_4}\qquad
\chemfig[baseline=26pt]{@{a}E^+}
\chemmove{
\draw[shorten <=2pt,shorten >=1pt](z).. controls +(-90:8mm) and +(-135:10mm)..(a);}
\arrow(@a--dd)[30]
\chemfig{R_2-C^+(-[:90]R_1)-C(-[:90]R_3)(-[:-90]E)-R_4}
\arrow(@a--ee)[-30]
\chemfig{R_2-C(-[:-90]E)(-[:90]R_1)-C^+(-[:90]R_3)-R_4}
\schemestop
\\\\
\schemestart
\chemfig{@{cc}{R_1}-[:-60]C(-[:-120]R_2)=[@{z}]C(-[:60]R_3)-[:-60]R_4}\qquad
\chemfig[baseline=26pt]{@{a}\charge{45[anchor=180+\chargeangle]=$\delta^+$}{Br}-@{x}\charge{45[anchor=180+\chargeangle]=$\delta^-$}{Br}}
\chemmove{
\draw[shorten <=2pt,shorten >=1pt](z).. controls +(-90:8mm) and +(-135:10mm)..(a);}
\arrow
\chemfig{R_2-C(-[:-90]Br)(-[:90]R_1)-@{r}C^+(-[:90]R_3)-R_4}
\+
\chemfig{@{q}\charge{145:2pt=$-$,-135=\:}{Br}}
\chemmove{
\draw[shorten <=2pt,shorten >=1pt](q).. controls +(-135:10mm) and +(-45:10mm)..(r);}
\arrow
\chemfig{R_2-C(-[:-90]Br)(-[:90]R_1)-C(-[:90]R_3)(-[:-90]Br)-R_4}
\schemestop\\\\

\schemestart
\chemfig{@{cc}{R_1}-[:-60]C(-[:-120]R_2)=[@{z}]C(-[:60]R_3)-[:-60]R_4}\qquad
\chemfig[baseline=26pt]{@{a}\charge{45[anchor=180+\chargeangle]=$\delta^+$}{H}-@{x}\charge{45[anchor=180+\chargeangle]=$\delta^-$}{X}}
\chemmove{
\draw[shorten <=2pt,shorten >=1pt](z).. controls +(-90:8mm) and +(-135:10mm)..(a);}
\arrow(@x--dd)[30]
\chemfig{R_2-C(-[:-90]H)(-[:90]R_1)-@{r}C^+(-[:90]R_3)-R_4}
\+
\chemfig{@{q}\charge{145:2pt=$-$,-135=\:}{X}}
\chemmove{
\draw[shorten <=2pt,shorten >=1pt](q).. controls +(-135:10mm) and +(-45:10mm)..(r);}
\arrow
\chemfig{R_2-C(-[:-90]H)(-[:90]R_1)-C(-[:90]R_3)(-[:-90]X)-R_4}
\arrow(@x--ee)[-30]
\chemfig{R_2-@{r}C^+(-[:90]R_1)-C(-[:-90]H)(-[:90]R_3)-R_4}
\+
\chemfig{@{q}\charge{145:2pt=$-$,-135=\:}{X}}
\chemmove{
\draw[shorten <=2pt,shorten >=1pt](q).. controls +(-135:10mm) and +(-45:10mm)..(r);}
\arrow
\chemfig{R_2-C(-[:-90]X)(-[:90]R_1)-C(-[:90]R_3)(-[:-90]H)-R_4}
\schemestop\\\\
\schemestart
\chemname{\chemfig{R-C^+(-[:90]H)-H}}{Primary Carbocation}\qquad \qquad
\chemname{\chemfig{R-C^+(-[:90]R)-H}}{Secondary Carbocation}\qquad \qquad
\chemname{\chemfig{R-C^+(-[:90]R)-R}}{Tertiary Carbocation}
\schemestop\\\\

\schemestart
\chemfig{@{cc}{R_1}-[:-60]C(-[:-120]R_2)=[@{z}]C(-[:60]R_3)-[:-60]R_4}\qquad
\chemfig[baseline=26pt]{@{a}\charge{45[anchor=180+\chargeangle]=$\delta^+$}{H}-\charge{45[anchor=180+\chargeangle]=$\delta^-$}{O}-S(=[:90]O)(=[:-90]O)-@{x}OH}
\chemmove{
\draw[shorten <=2pt,shorten >=1pt](z).. controls +(-90:8mm) and +(-135:10mm)..(a);}
\arrow(@x--dd)[30]
\chemfig{R_2-C(-[:-90]H)(-[:90]R_1)-@{r}C^+(-[:90]R_3)-R_4}
\+
\chemfig{@{q}\charge{145:2pt=$-$,-135=\:}{O}-S(=[:90]O)(=[:-90]O)-OH}
\chemmove{
\draw[shorten <=2pt,shorten >=1pt](q).. controls +(-135:10mm) and +(-45:10mm)..(r);}
\arrow
\chemfig{R_2-C(-[:-90]H)(-[:90]R_1)-C(-[:90]R_3)(-R_4)-[:-90]O-[:-45]S(=[:45]O)(=[:-135]O)-[:-45]OH}
\arrow(@x--ee)[-30]
\chemfig{R_2-@{r}C^+(-[:90]R_1)-C(-[:-90]H)(-[:90]R_3)-R_4}
\+
\chemfig{@{q}\charge{145:2pt=$-$,-135=\:}{O}-S(=[:90]O)(=[:-90]O)-OH}
\chemmove{
\draw[shorten <=2pt,shorten >=1pt](q).. controls +(-135:10mm) and +(-45:10mm)..(r);}
\arrow
\chemfig{R_2-C(-[:-90]O(-[:-135]S(=[:-45]O)(=[:135]O)(-[:-135]OH)))(-[:90]R_1)-C(-[:90]R_3)(-[:-90]H)-R_4}
\schemestop\\\\

\schemestart
n
\chemfig[baseline=-26pt]{@{cc}{R_1}-[:-60]C(-[:-120]R_2)=[@{z}]C(-[:60]R_3)-[:-60]R_4}
\arrow
\chemfig{\vphantom{CH_2}-[@{op,.75}]C(-[:90]R_1)(-[:-90]R_2)-C(-[:90]R_3)(-[:-90]R_4)-[@{cl,0.25}]}
\polymerdelim[height = 35pt, depth = 40pt, indice = \!\!n]{op}{cl}
\schemestop
\schemestart
n
\chemfig[baseline=-26pt]{@{cc}{Cl}-[:-60]C(-[:-120]H)=[@{z}]C(-[:60]H)-[:-60]H}
\arrow
\chemfig{\vphantom{CH_2}-[@{op,.75}]C(-[:90]Cl)(-[:-90]H)-C(-[:90]H)(-[:-90]H)-[@{cl,0.25}]}
\polymerdelim[height = 35pt, depth = 40pt, indice = \!\!n]{op}{cl}
\schemestop\\\\\\\\
\schemestart
\chemname{\chemfig{R-C(-[:-90]OH)(-[:90]H)-H}}{Primary Alcohol ($^\circ1$)}\qquad \qquad
\chemname{\chemfig{R-C(-[:-90]OH)(-[:90]R)-H}}{Secondary Alcohol ($^\circ2$)}\qquad \qquad
\chemname{\chemfig{R-C(-[:-90]OH)(-[:90]R)-R}}{Tertiary Alcohol ($^\circ3$)}
\schemestop\\\\
Back to eliminating stuff? jk this is E\_2\\\\

\schemestart
\chemfig{R_1-C(-[:90]H)(-[:-90]@{a}\charge{0=\:}{O}-[:-90]H)(-C(-[:90]R_2)(-[:-90]H)(-R_3))} \qquad
\chemfig{@{b}H^+}
\chemmove{
\draw[shorten <=2pt,shorten >=1pt](a).. controls +(0:8mm) and +(-135:8mm)..(b);}
\arrow
\chemfig{R_1-C(-[:90]H)(-[@{c}:-90]@{d}O^+(-[:-150]H)(-[:-30]H))(-C(-[:90]R_2)(-[:-90]H)(-R_3))}
\chemmove{
\draw[shorten <=2pt,shorten >=1pt](c).. controls +(0:6mm) and +(30:4mm)..(d);}
\arrow
\chemfig{R_1-C^+(-[:90]H)(-[@{e}]C(-[:90]R_2)(-[@{f}:-90]H)(-R_3))}
\chemmove{
\draw[shorten <=2pt,shorten >=1pt](f).. controls +(180:4mm) and +(-90:4mm)..(e);}
\+
\chemfig{H_2O}
\arrow
\chemfig{H-[:-60]C(-[:-120]R_1)=C(-[:60]R_2)-[:-60]R_3}
\arrow{0}[,0]\+
\chemfig{H_2O}
\+
\chemfig{H^+}
\schemestop\\\\


\end{document}
